\documentclass[10pt,a4paper]{scrartcl}

\usepackage[T1]{fontenc}
\usepackage[ngerman]{babel}
\usepackage{graphicx}
\usepackage{paralist}
\usepackage{fontspec} 
\usepackage{xcolor} 
\usepackage[unicode=true]{hyperref}
\hypersetup{pdfborder={0 0 0}, breaklinks=true}

\defaultfontfeatures{Mapping=tex-text}
% \setmainfont{IBMPlexSans-Light}[
%   BoldFont    = IBMPlexSans-Medium,
%   ItalicFont  = IBMPlexSans-LightItalic,
%   SlantedFont = IBMPlexSans-LightItalic
% ]
\setmainfont[Path=/System/Library/Fonts/]{Palatino.ttc}
\setsansfont[Path=/Library/Fonts/]{Futura.ttc}
\setmainfont{ztm-Reg.otf}
\setsansfont{zhv-Reg.otf}



\typearea{12}
\pagestyle{headings}

\newcounter{zwei}
\setcounter{zwei}{2}

\newcommand{\mobilnummer}{0\,157\,/\,74\,72\,71\,42}

\newcommand{\unterschrift}{\vspace{5ex plus1ex minus1ex}%
  \includegraphics[width=6cm]{unterschrift_farbe.png}\\
  Alexander Krumeich\\Köln, den \today}


\setlength{\parindent}{0mm}
\renewcommand{\baselinestretch}{1.05}

\hyphenation{Krum-eich}

\pagestyle{myheadings}
\markright{\hfill\textnormal{Alexander Krumeich}}
\thispagestyle{empty}

\makeatletter
\def\@evenfoot{}
\def\@oddfoot{}
\makeatother


% ----------------------------------------------------------------------------

\begin{document}

\thispagestyle{empty}

\hfill{\textsf{\Large{}Alexander Krumeich}}\\
\hrule
\vspace*{2mm}
\textsf{Niehler Str.~1,\enskip{}50670~Köln\hfill{} alexander@krumeich.de\hfill{}
\mobilnummer{}}
\vspace*{5mm}

% ----------------------------------------------------------------------------

\section*{Fotoaufnahmen von Kindern}

\subsection*{Erläuterung des Vorhabens}


Ich beabsichtige, als Privatperson Fotoaufnahmen der Kinder der
Schildkrötenklasse (1b/2b) und der Mäuseklasse (3b/4b) der GGS Balthasarstraße
zu machen. Die Fotos sollen Ihnen (Eltern, Kinder, Klassenlehrerinnen und
OGS-Betreuer\-Innen) als Erinnerungsstücke unentgeltlich zugänglich gemacht
werden. Darüber hinaus finden die Fotos keine Verwendung.

Die Fotos sollen auf meiner privaten Website \url{https://krumeich.de/} für
einen begrenzten Zeitraum von ca. 6~Wochen passwortgeschützt abrufbar sein. Die
Fotos werden in dieser Zeit in Form einer Galerie zum Anschauen bereitstehen und
zusätzlich zum Download verfügbar sein. Die Fotos werden nicht in sozialen
Medien publiziert oder mit Messenger-Diensten versendet.

Die Website ist über das Internet grundsätzlich öffentlich erreichbar.  Die
Galerie und die Dateien zum Download werden in einem passwortgeschützten Bereich
liegen. Die (nicht-individuellen) Zugangsdaten werden Ihnen mit der
Ankündigung der Veröffentlichung der Fotos zugesendet. Darüber hinaus wird der
Bereich der Website, in dem die Fotos liegen, nicht von anderen Seiten
verlinkt. Dadurch ist für den Abruf der Fotos die Kenntnis der genauen Adresse
notwendig.

Trotz dieser Vorsichtsmaßnahmen kann ich nicht ausschließen, dass beliebige
Personen die Bilder ansehen oder herunterladen. Meine Website speichert keine
Benutzerdaten, somit ist für mich nicht nachvollziehbar, wer die Bilder ansieht
oder herunterlädt.

\subsection*{Einwilligung des/der Personensorgeberechtigten}

Mit Ihrer Unterschrift bestätigen Sie Ihr Einverständnis, dass ich Ihr Kind
einzeln, in Kleingruppen und im Klassenverband fotografiere und die Fotos wie
oben beschrieben veröffentliche. Gleichzeitig bestätigen Sie, dass Sie mit Ihrem
Kind besprochen haben, dass es fotografiert wird und die Bilder veröffentlicht
werden. Sie bestätigen, dass auch Ihr Kind damit einverstanden ist.

Sie bestätigen ebenfalls, dass Sie selbst die Fotos und die Zugangsdaten nicht
weiter verbreiten, in sozialen Medien publizieren oder mit Messenger-Diensten
versenden, ohne dass Ihnen eine Einverständniserklärung der jeweiligen
Personensorgeberechtigten vorliegt.

\vspace{2ex}

Bitte kontaktieren Sie mich bei Rückfragen. Vielen Dank für Ihr Verständnis.

\vspace{1cm}

Name und Klasse des Kindes: ~\hrulefill{}



\vspace{1.5cm}

~\hrulefill{}

\begin{minipage}{10cm}
  \vspace{1ex}\par{}
  Ort, Datum, Unterschrift des/der Personensorgeberechtigten\par{}
  \vspace{1ex}
  (Bei Fotoaufnahmen minderjähriger Kinder sind die Unterschriften aller
  Personensorgeberechtigten notwendig.)
\end{minipage}



\end{document}



% Local Variables:
% TeX-engine: luatex
% TeX-master: t
% End:
